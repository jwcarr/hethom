\documentclass[doc,biblatex]{apa7}
\DeclareLanguageMapping{american}{american-apa}
\addbibresource{references.bib}
\captionsetup[figure]{textfont=rm,font=small}
\usepackage{amsmath}
\usepackage{fontspec}
\setmainfont{Times New Roman}


\title{How does the writing system get informative? Differentiation or conservation?}

\shorttitle{Informativity in the writing system}

\authorsnames[1,1]{Jon W. Carr, Kathleen Rastle}

\authorsaffiliations{{Department of Psychology, Royal Holloway, University of London, England}}

% \authornote{\textbf{Cite as:} Carr, J.\@ W., \& Rastle, K.\@ (2023). Informativity in the writing system. \textit{OSF Preprints}. Version~0.}

\abstract{Abstract.}

\keywords{communication; cultural evolution; informativity; iterated learning; language evolution; orthography; spelling; sound change; writing}

\begin{document}

\maketitle

\noindent
Written languages represent spoken languages. However, there are many ways in which written languages diverge. E.g. spacing between words helps readers parse a sentence even if they are not technically required. Written languages can also have there own regularities such as spelling related words similarly (magician), past tense spellings directly conveying meaning, certain spellings conveying word class ous. These features abound in writing, especially in languages such as English that have not been subject to serious regulation.

One notable case of this is the heterographic homophones---words that sound alike but which are written with different spellings. For example, <meat> and <meet> are both pronounced /miːt/. Furthermore, these spellings can be cognitively useful . For example, faced with a sentence beginning /ðɛr/, a listener will have high uncertainty about what word---or even what sentence structure---is likely to come next: a noun, as in /ðɛr dɒg/, a form of the verb to be, as in /ðɛr ɪz/, or the progressive form of a verb, as in /ðɛr gəʊɪŋ/. In writing, by contrast, this uncertainty is greatly reduced; the spellings <their>, <there>, and <they're> differentiate between these cases, allowing the reader to get directly to the meaning. This idiosyncratic spelling of words that are homophonous in speech may confer certain benefits in reading, particularly in terms of comprehension. Heterography permits a writing system to be---at least in some areas---more informative than its spoken counterpart.

However, this additional informativeness comes at a cost. A system of words that conveys more information will---in general---be more complex than one that conveys less information. There exists a tradeoff between the simplicity of the spelling system and its ability to be expressive. A simple spelling system is one that is easy to learn, use, and process; for example, by being transparent with respect to phonology. The idea of a tradeoff between simplicity and informativeness in language structure has a longstanding history \parencite{Gabelentz:1891, Zipf:1949, Martinet:1952, Rosch:1978} and has recently been explored in a wide range of both typological \parencite{KempRegier:2012, Kemp:2018} and experimental \parencite{Carr:2020, Kirby:2015} studies. Of particular note here is the idea that the pressure for simplicity in language structure derives principally from learning, while the pressure for informativeness derives from the need for precise communication.

Any explanation for heterographic homophones must, therefore, demonstrate that heterographic spellings convey a benefit in terms of informativeness that is outweighed by the cost of learning the (often arbitrary) spelling distinction. What are the benefits? There seem to be broadly two kinds of benefit. (1) disambiguation in the writing of words that would otherwise be ambiguous if spelled transparently. (2) permitting the cognitive system to rapidly and accurately attain the meaning of a word. In this paper we limit ourselves to informativeness of the first kind, which is more amenable to experimental testing.

Is it possible, then, that heterography may be selected for in the cultural evolution of orthography? And if so, what might be the mechanism through which such selection could operate? \textcite[pp.~325--326]{Berg:2021} sketch two models of the emergence of heterography (see Fig.~\ref{}). The first model, \textit{the differentiation model}, explains heterography through the creation of a new orthographic form to differentiate between homophones. Imagine, for example, that English had chosen to adopt the French spelling <banque> for banks of the financial kind to differentiate them from banks of the river kind, resulting in a state of heterography: /baŋk/–/baŋk/ in speech, but <bank>–<banque> in writing. The second model, \textit{the conservation model}, explains heterography as the historical residue of sound change: Two spoken forms merge and become homophonous, but their heterographic spellings are conserved in the orthography. For example, the words \textit{night} and \textit{knight}---heterographic homophones in modern English---were previously pronounced differently from each other (in Old and Middle English, /kn/ was a permissible consonant cluster, as it still is in modern Dutch and German).

byte bite

bizarre bazaar

bombe

cache

<cheque>, which the OED describes as a ``a differentiated spelling of \textit{check}.''

Tindr


Innovation

French spellings
double letters inn
add -e


Processes

Sound change
dialect merger (bury is the spelling from one dialect with the pronunciation from another dialect)
Borrowing (English tends to retain spelling from the source language)
Printers innovations
- Etymological spellings (doubt)
- Deliberate design of heterographs
- Morphological spellings


%~%~%~%~%~%~%~%~%~%~%~%~%~%~%~%~%~%~%~%~%~%~%~%~%~%~%~%~%~%~%~%~%~%~%~%~%~%~%~%~%~%~%~%~%~%~%~%~%~%~%~%

\section{Experiment 1}

Our first experiment tests the ability of the differentiation model to explain the emergence of informativeness in orthography. In particular we had two main hypotheses:

variation

communication

Preregistered \url{https://aspredicted.org/YLG_CKD}

\subsection{Methods}

To simulate the cultural evolution of orthography, we adopt the experimental iterated learning paradigm \parencite{Kirby:2008, Kirby:2015}. In this paradigm, some linguistic system is passed along a \textit{transmission chain} of human participants through repeated learning and production. Participant $i$ in the transmission chain learns the system based on the linguistic output of participant $i-i$ and then produces new linguistic output for participant $i+1$ to learn from. Through several \textit{generations} of this process, the linguistic system can gradually adapt to the learning biases of the human learners, yielding interesting emergent phenomena, such as compositionality \parencite{Kirby:2008, Kirby:2015}, combinatoriality \parencite{Verhoef:2015}, category structure \parencite{Carr:2017, Carr:2020}, and regularization \parencite{Smith:2010, Ferdinand:2019}.

To explore the hypotheses outlines above, we conducted the experiment under our different conditions: high variation vs.\@ no variation and communication vs.\@ no communication.

\subsubsection{Participants}

We recruited 539 participants via the Prolific platform. Participants were paid £3.00 for participation plus additional bonuses of up to £1.36 as detailed below (median bonus: £0.xxx). Participants were required to have English as a first or second language to ensure they could understand the consent form and instructions; no other restrictions were applied. Participants' most common first languages were: xxx. 44 participants (8\%) were lost to rejections or communication-game matching failures, leaving us with 495 participants in our final dataset.

\subsubsection{Stimuli}

Participants were taught words for 16 ``alien'' objects---four shapes that could appear in four colors (see Fig.~\ref{}). The alien words had both a spoken and a written form and were composed of a stem and a suffix. The stems, which express the shape dimension, were /buv/ <buv>, /zɛt/ <zet>, /gaf/ <gaf>, and /wɒp/ <wop>. To make the stems easy to learn, they were designed to be graphically and phonetically iconic of the shapes they represent. The suffix was always pronounced /ikəʊ/ in speech, but its spelling was free to evolve over time. Thus, the spoken form of the language consists of just four unique words---/buvikəʊ/, /zɛtikəʊ/, /gafikəʊ/, and /wɒpikəʊ/---that mark a shape distinction; however, the \textit{spelling} of the suffix could potentially take on different forms to mark color. The spoken forms were synthesized using the Apple text-to-speech synthesizer (the Tessa voice).

The transmission chains were seeded with an orthographic system that may exhibit high variation or no variation according to condition. In the no-variation conditions, the suffix had only one spelling, although each chain was initialized with a different randomly generated spelling. To generate a spelling we randomly choose a grapheme from each of the sets \{<ee>, <i>, <y>\}, \{<c>, <ck>, <k>\}, and \{<oe>, <oh>, <ow>\} and concatenate them. For example, in one chain the suffix might initially be spelled <icoe>, while in another chain it might initially be spelled <yckow>. In the high-variation conditions, the spellings were randomly generated not just for each chain but also for each of the 16 objects. Thus, although the star shaped objects are all called /zɛtikəʊ/ in speech, the different colors will all have different random spellings (e.g., <zetikow> for pink, <zeticoe> for gray, <zeteeckoe> for blue, and <zetyckoh> for yellow). These spellings were not systematic, however; the <yckoh> spelling was not systematically used to represent yellow across different shapes. Essentially, then, in the no-variation conditions there is initially only one possible way to spell /ikəʊ/, while in the high-variation conditions there are many possible ways to spell it without any systematic pattern.

\subsubsection{Transmission procedure}

As outlined above, the participants were arranged into transmission chains such that the spellings produced by one participant would subsequently be taught to the next participant in the chain. The first participant in a chain was taught the randomly generated orthographic system explained in the previous section, and this system was then free to evolve as it was culturally transmitted. This was subject to a bottleneck on transmission: Not all 16 spellings were transmitted from one generation to the next; rather, the participant at generation $i$ would observe only 12 of the 16 spellings generated at generation $i-1$ (at least one of each shape and at least one of each color). This procedure continued until the chain converged on a particular orthographic system. A chain is said to converge when two consecutive generations produce identical spellings for all 16 items, suggesting that the chain is unlikely to experience significant further change. We adopted this convergence-based approach because we were interested in revealing what kinds of orthographic system are stable under the four different conditions.

\subsubsection{Training procedure}

Participants were trained on the spoken and written forms through a combination of passive exposure trials and mini-test trials. In the passive exposure trials, the alien object appeared first, accompanied by the written and spoken forms after a 500~ms delay. In the mini-test trials, which were interleaved among the passive exposure trials, the participant had to type in the appropriate written form for one of the objects. The participant could freely type in a word and would then be given feedback on any spelling errors (additions shown with green text and deletions shown with red strikethrough text). Participants received a 1p bonus for spelling the stem correctly or a 2p bonus for spelling the entire word correctly.

\subsubsection{Test procedure}



\subsubsection{Communication game procedure}





\subsection{Results}

Learnability, measured as transmission error (lower error = higher learnability). Mean Levenshtein edit distance between the words produced at generation i and the corresponding words produced at generation i-1.

Simplicity, measured as complexity (lower complexity = higher simplicity). The length (in bits) of the shortest grammar that losslessly compresses the lexicon, as computed by the Grammarette package: https://github.com/jwcarr/grammarette. Finding the shortest grammar for some lexicon is a non-trivial problem, and it is likely that we will need to adapt the method in response to the collected data.

Communicative cost is given by
\begin{equation}
\mathrm{cost}(L) := \frac{1}{|U|} \sum_{m \in U} -\log \mathrm{Pr}(m|s_m),
\end{equation}
where $\mathrm{Pr}(m|s_m)$ is the probability that a listener would infer meaning $m$ given that a speaker produced signal $s$ for meaning $m$. Here we assume this probability to be $1/|M_s|$, where $M_s$ is the set of meanings labeled $s$.

\subsection{Summary}

%~%~%~%~%~%~%~%~%~%~%~%~%~%~%~%~%~%~%~%~%~%~%~%~%~%~%~%~%~%~%~%~%~%~%~%~%~%~%~%~%~%~%~%~%~%~%~%~%~%~%~%

\section{Experiment 2}

Our first experiment tests the ability of the conservation model to explain the emergence of informativeness in orthography.

\subsection{Methods}

The methods were mostly identical to Experiment~1 with three major exceptions.

sound change

non convergence

no variation

compositional start

\subsubsection{Participants}

\subsubsection{Stimuli}

The alien objects and word stems were identical to Experiment~1. Unlike Experiment~1, however, the chains were initialized with a fully compositional language that uses four distinct suffixes to systematically express each of the four colors. The suffixes took the form /ɪ\textit{CV}/ where $C$ is a voiceless fricative consonant from the set \{/θ/, /s/, /ʃ/, /h/\} and $V$ is a vowel from the set \{/ə/, /ɛɪ/, /əʊ/, /u/\}. The consonant and vowel sets were randomly permuted for each iterated learning chain, such that the suffix forms were unique to each chain. For example, under the permutations given above, the resulting suffixes would be /ɪθə/, /ɪsɛɪ/, /ɪʃəʊ/, and /ɪhu/, but another iterated learning chain might use the suffixes /ɪsəʊ/, /ɪʃɛɪ/, /ɪθu/, and /ɪhə/. The initial orthographic system used to seed the chain was generated based on the following phoneme–grapheme mapping \{/θ/→<th>, /s/→<s>, /ʃ/→<sh>, /h/→<x>, /ə/→<a>, /ɛɪ/→<e>, /əʊ/→<o>, /u/→<u>\}. The suffixes were designed to be distinctive (and therefore easy to memorize and recall), but also similar enough to (mostly) allow for somewhat plausible sound mergers and result in somewhat plausible spellings following sound merger (e.g., it is plausible that /θ/ might be supplanted by /s/ in speech or that /θ/ might be spelled <x> in writing). We attempted to achieve this balance by combining consonants that are very similar (differing only in place of articulation) with vowels that are very dissimilar.

\subsubsection{Sound change}

Each iterated learning chain was run for 12 generations, divided into four epochs of three generations. In the first epoch, all suffixes were distinct, allowing the spoken language to express all four colors, but by the fourth epoch all suffixes were reduced into a single phonetic form, making the spoken language entirely uninformative about color. This was achieved through three sound changes occurring between the four epochs. In the first sound change, two of the suffixes were chosen at random to be merged. To perform the merger, the consonant from one of the suffixes (chosen at random) was paired with the vowel from the other suffix, resulting in a new suffix form that would replace the original two. For example, assuming the language has the suffixes /ɪθə/ (pink), /ɪsɛɪ/ (gray), /ɪʃəʊ/ (blue), and /ɪhu/ (yellow), we might merge /ɪsɛɪ/ and /ɪhu/ into /ɪhɛɪ/, such that the spoken language is no longer able to mark a distinction between gray and yellow. In the second sound change, the same procedure is applied to the two suffixes that were not selected previously (e.g., /ɪθə/ and /ɪʃəʊ/ become /ɪθəʊ/). And in the third and final sound change, the two suffix forms created through the first and second sounds changes---/ɪhɛɪ/ (grey/yellow) and /ɪθəʊ/ (pink/blue)---merge into a single form (e.g., /ɪθɛɪ/). Importantly, the spellings do not change automatically following a sound change event; instead, the orthographic system is free to adapt (or not) in response to sound changes.

%    ORIG.    SC1      SC2      SC3

% P: /ɪθə/           > /ɪθəʊ/ > /ɪθɛɪ/
% G: /ɪsɛɪ/ > /ɪhɛɪ/          > /ɪθɛɪ/
% B: /ɪʃəʊ/          > /ɪθəʊ/ > /ɪθɛɪ/
% Y: /ɪhu/  > /ɪhɛɪ/          > /ɪθɛɪ/

\subsection{Results}



\subsection{Summary}

%~%~%~%~%~%~%~%~%~%~%~%~%~%~%~%~%~%~%~%~%~%~%~%~%~%~%~%~%~%~%~%~%~%~%~%~%~%~%~%~%~%~%~%~%~%~%~%~%~%~%~%

\section{Discussion}

%~%~%~%~%~%~%~%~%~%~%~%~%~%~%~%~%~%~%~%~%~%~%~%~%~%~%~%~%~%~%~%~%~%~%~%~%~%~%~%~%~%~%~%~%~%~%~%~%~%~%~%

\section{Acknowledgments}

\noindent This work was funded by a grant from the Leverhulme Trust.

\printbibliography

\end{document}